Spectrum Sensing is a key technology for Cognitive Radio: the initial task of any cognitive device will be to accurately sense and classify spectral bands. The sampling rates required by many bands (for example TV Whitespaces) makes spectrum sensing costly with current technology. Currently many spectra are sparse and this sparsity can be exploited by using the modern paradigm of Compressive Sampling. This thesis explores two methods to reduce the sampling rate per device.

The first is to use a distributed network of sensors to reduce the sensing load. The second is to avoid reconstruction of the spectrum itself, but to infer the occupied bands directly from the compressive measurements. In other words we do not reconstruct the spectrum as an intermediate step for classification.

One problem with current approches to spectrum recovery is that should TVWS bands come into use, they will not remain sparse. Instead of seeking a sprase gradient we require sparsity in the gradient of the spectrum. \frac{•}{•}We develop a model of the spectral gradient which allows reconstruction from Compressive measurements. Further, we develop a distributed ADMM algorithm for the LASSO with exact, closed form, expressions for the minima per iteration, reducing the computational cost of the algorithm. This allows us to perform spectral reconstruction which is blind to statistics such as the signal sparsity, using only simple statistical operations such as least squares and shrinkage. Numerical experiments show that the algorithm performs to within \(10^{-2}\) of an equivalent centralised algorithm. The algorithm displays the rapid convergence of ADMM, but with a substantially reduced computational burden.

We consider the group testing problem, in the case where the items are defective independently but with non-constant probability.
We introduce and analyse an algorithm to solve this problem by grouping items together appropriately. We give conditions under which  the
algorithm performs essentially optimally in the sense of information-theoretic capacity.
This has applications to the allocation of spectrum to cognitive radios, in the case where a database gives prior information that a particular band will be occupied.