\documentclass{article}
\bibliographystyle{plain}
\linespread{1}
\usepackage{wrapfig}
\usepackage{amsfonts}
\usepackage{amsmath}
\usepackage[utf8]{inputenc}
\usepackage[T1]{fontenc}
\usepackage{graphicx}
%\usepackage[english]{babel}
\usepackage[algoruled]{algorithm2e}

\renewcommand{\theequation}{\thesection.arabic{equation}}

\renewcommand{\thefigure}{\thesection.\arabic{figure}}



\renewcommand{\vec}[1]{\mathbf{#1}}
\renewcommand{\theequation}{\thesubsection.\arabic{equation}}
\DeclareGraphicsExtensions{.pdf,.png,.jpg, .gif}

\usepackage{amsthm}

\usepackage{mathtools}

%\usepackage[OT2,T1]{fontenc}
%\DeclareSymbolFont{cyrletters}{OT2}{wncyr}{m}{n}
%\DeclareMathSymbol{\sha}{\mathalpha}{cyrletters}{"58}

\DeclareFontFamily{U}{wncy}{}
\DeclareFontShape{U}{wncy}{m}{n}{<->wncyr10}{}
\DeclareSymbolFont{mcy}{U}{wncy}{m}{n}
\DeclareMathSymbol{\Sh}{\mathord}{mcy}{"58} 
\DeclareMathOperator*{\argmin}{arg\,min}
\DeclareMathSymbol{\kronecker}{\otimes}

\newcounter{eqn}
\renewcommand*{\theeqn}{\alph{eqn})}
%\newcommand{\num}{\refstepcounter{eqn}\text{\theeqn}\;}

\makeatother
\newcommand{\vectornorm}[1]{\left|\left|#1\right|\right|}
\newcommand*\conjugate[1]{\bar{#1}}

\theoremstyle{plain}
\newtheorem{thm}{Theorem}[chapter]
\newtheorem{defn}{Definition}
 %\theoremstyle{plain}
  \newtheorem{theorem}{Theorem}[section]
  \newtheorem{corollary}[theorem]{Corollary}
  \newtheorem{proposition}[theorem]{Proposition}
  \newtheorem{lemma}[theorem]{Lemma}
\newtheorem{example}[theorem]{Example}
  \newtheorem{definition}[theorem]{Definition}
  \newtheorem{conj}[theorem]{Conjecture}
 \newtheorem{condition}{Condition}
 \newtheorem{remark}[theorem]{Remark}

\newcommand{\supp}{\operatorname{supp}} 
\newcommand{\vc}[1]{{\mathbf{ #1}}}
\newcommand{\tn}{\widetilde{\nabla}_{n} }
\newcommand{\Z}{{\mathbb{Z}}}
\newcommand{\re}{{\mathbb{R}}}
\newcommand{\II}{{\mathbb{I}}}
\newcommand{\ep}{{\mathbb{E}}}
\newcommand{\pr}{{\mathbb{P}}}
\newcommand{\FF}{{\mathcal{F}}}
\newcommand{\TT}{{\mathcal{T}}}
\newcommand{\phin}{\phig{n}}
\newcommand{\phig}[1]{\phi^{(#1)}}
\newcommand{\ol}[1]{\overline{#1}}
\newcommand{\eff}{{\rm eff}}
\newcommand{\suc}{{\rm suc}}
\newcommand{\tends}{\rightarrow \infty}
\newcommand{\setS}{{\mathcal{S}}}
\newcommand{\setP}{{\mathcal{P}}}
\newcommand{\setX}{{\mathcal{X}}}
\newcommand{\nec}{{\rm nec}}
\newcommand{\bd}{{\rm bd}}
\newcommand{\kronecker}{{\otimes}}

\DeclareUnicodeCharacter{00A0}{~}

\title{Introduction}
\author{Tom Kealy}

\begin{document}
\maketitle

\section{Introduction}

In order to meet exponentially growing consumer demand for wireless data, radio spectrum regulators considering opportunistic spectrum access policies. Historic spectrum regulation focussed on exclusive frequency assignments (licensing), with spatial and frequency separation to mitigate interference between users. However, this approach leads to considerable underutilisation in both space and time. Thus, faced with a need to provide 1000 times the bandwidth in 10 years, regulators are considering agile access technologies on a licence exempt basis.

Before opportunistic access is a reality, speedy, robust, and accurate estimation of frequency spectra must be made. This is a challenging statistical and engineering problem, limited by characteristics of wireless channels such as multipath fading, and shadowing. The proposed bands have a large bandwidth, containing sub-channels which are not contiguous but statistically correlated, and radio-wave fading environment which can mask high-powered transmissions. Traditional methods are un-viable in these conditions; either requiring expensive hardware to meet the data rate required to perform the sampling, or a large number of RF components to turn a single wideband channel into many narrowband ones.

This project addresses the issue of estimating available frequencies for opportunistic transmission, from a set of underdetermined measurements. The compete set of measurements may be available to a particular sensor. Or, the measurements may be distributed over a network of sensors, improving estimation accuracy, and 

The growing number of wireless devices is placing increasing demand on radio spectrum. Consumers are demanding faster speeds and better quality connections in more places. However, there is a limited amount of frequencies to transmit on. Consequently, demand for frequencies which provide sufficient bandwidth, good range and in-building penetration is high.

Not all frequencies are used at all times and in all places. Judicious spectrum management could alleviate the issue of too few frequencies being available to devices by developing approaches to interleaving opportunistic transmissions within established bands. 

There are benefits to spectrum sharing beyond simply satisfying consumer demand. Innovative wireless applications such as wireless rural broadband, remote monitoring, and machine-to-machine applications will be made viable without the need to purchase exclusive access to a specific frequency. 

The historic allocation process has placed unused spectrum between adjacent radio channels, in an attempt to avoid interference between consecutive users. However, technical changes in transmission standards are making some bands available as modern coding and modulation techniques are more spectrally efficient. In particular, in the UK, the switchover from analogue to digital terrestrial broadcast has freed many previously used bands.

Recent regulatory focus has been on frequencies in the TV broadcast bands 470-790MHz. In the UK, TV channels are broadcast using up to six multiplexes, each requiring an 8MHz channel. A total of 32 of these channels are allocated for TV broadcast, whilst only 6 of these channels are required to receive the 6 multiplexes at any given location. This is because TV broadcast is high powered and needs spatial separation between coverage areas to avoid interference. As such, the majority of TV frequencies reserved for TV broadcast are unused in any given place. These white spaces are spectrum which has been left over to prevent interference between primary users (such as TV broadcast). These frequencies can be used on an opportunistic basis by relatively low-powered devices. Whilst current focus is on the TVWS bands, the work presented in this thesis can be used in other, as yet unrealised frequencies.

Currently access to spectrum is managed in two ways: licensed and license exempt access. Licensing authorises a particular user/users to exclusive use of a specific frequency band. License exempt access allows any user to access a band, provided they meet technical requirements intended to limit interference in on other users. This is a particularly pressing issue for the co-existence of licensed users and licence exempt users in the same band.

Devices seeking to access white spaces need a robust mechanism for learning which frequencies can be used at a particular time and location. The approach currently being taken by OFCOM and the FCC is to maintain a set of databases, which map the location of white spaces based on knowledge of existing spectrum users. An alternative approach is for devices to monitor the use of spectrum individually.

One approach to white-space access is to maintain a database of currently available frequencies. The database would contain up to date information about incumbents, including television transmitters and wireless microphones. It would also need to maintain information about currently operative secondary users. Devices would register with the database, based on their geo-location and the service would determine availability for the device, either based on in situ measurements, or from a propagation model.

Location based measurements are a costly approach, as they need to be redone every time primary user transmission characteristics change, and they cannot be done economically in difficult to reach places. Propagation models can achieve good accuracy, but at the cost of being complex and computationally intensive. Simplistic models do not agree with ground truth measurements. 

Devices must also supply geographic information to a database service. How this is done has yet to be codified, but current methodologies all have drawbacks. GPS positioning works well outdoors, but is inaccurate indoors. Cellular location can yield errors of up to a mile. Using the base-station location as a proxy for the device location results in an unacceptably high loss of whitespaces, as decisions would have to be needlessly conservative.

An alternative to maintaining a database of available frequencies, is for secondary users to independently sense spectrum. This method is subject to a number of technical limitations. The proposed bands have an ultra-wide bandwidth. Traditional sensing approaches to this have been either to divide the band into a number of contiguous narrow bands, or to simply use a high rate processing device to capture the necessary samples. Both methods are prohibitively expensive, and require dedicated and energy hungry hardware. Further, traditional statistical spectral detection and estimation techniques make assumptions about the primary users signal. Such assumptions include cyclo-stationarity, the presence of specific waveform patters, the use of known pilot sequences, and sufficiently high powered transmissions that energy detection is viable. In practice, some (or all) of these assumptions are violated; either because assumed patterns are not present in the  transmission, or because the radio environment has a deep fade between the primary and secondary user masking high powered transmissions.

The biggest issues for single node sensing are multipath fading, shadowing, and PU receiver uncertainty. Cooperation between a network of nodes can improve sensing performance through spatial diversity. In cooperative sensing information from geographically diverse nodes is aggregated in the decision making process. Combining observations can overcome the deficiencies of observations unique to individual cognitive radios. Spatial diversity makes it unlikely that all radios will experience the same fading and receiver uncertainty. Approaches to cooperative sensing based on Nyquist sensing typically involve energy detection at each node, along with a gain combining procedure (which is either performed at a fusion node, or via a decentralised algorithm). Cooperative sensing can use such simple methods at each node as receiver sensitivity can be mitigated by using multiple statistically independent observations. This reduces the complexity and cost of the individual cognitive radios.

Cooperative sensing has drawbacks however. Spatial diversity does not necessarily mean statistically independent observations. A subset of nodes all blocked by the same object will make measurements which are correlated and corrupted by the same shadowing for example. Cooperative sensing also involves an overhead with the being part of a network. This is any extra time, energy and processing when compared to sensing individually. 

An alternative sensing strategy is compressive sampling (CS). This is a new paradigm in signal processing which has emerged over the last decade, and has had significant success in imaging problems. In particular compressive sampling strategies have reduced the time needed for an MRI scan from 2 minutes to 30 seconds. Patients previously had to hold their breath for the entire duration. The central innovation in compressive sensing is that randomness is an effective strategy for sensing sparse signals. In practice much of the TVWS spectrum is unoccupied, and CS can leverage this sparsity to reduced the sampling rate of the CRS.

Group Testing is a model sparse inference problem, first introduced by Dorfmann in the context of testing for rare disease. Given a population of items, some small fraction of which has an interesting property (labelled defective), Group Testing proposes algorithms to find those items efficiently. What allows these items to be discovered, in fewer tests than the total number of items, is testing pools of items. That is, items aren't tested individually, but several together. In testing for rare diseases the blood samples are mixed and the mixture is tested. This allows fast elimination of non-defective items. Popular pooling designs are randomised - items are included in a test with independent probabilities. We consider probabilistic algorithms with non uniform probabilities of each item's defectivity. These non-uniform priors could come from a TVWS database of possibly occupied spectrum, or from summarising risk and family history in disease testing.

The structure of this thesis is as follows. Chapter 2 covers the relevant material on Nyquist sensing theory and Nyquist approaches to the spectrum sensing problem. It covers energy detection, feature detection, matched filtering, and cooperative approaches to the problem.

Chapter 3 covers the theory of compressive sensing and 

Chapter 4 covers the theory of ADMM and optimisation on graphs.

Chapter 5 covers our approach to solving the optimisation problem.

Chapter 6 Covers the model we used.

Chapter 7 covers

Chapter 8 covers group testing.



\end{document}