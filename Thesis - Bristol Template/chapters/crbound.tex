\documentclass[12pt, a4paper, titlepage]{article}
\linespread{1}
\usepackage{wrapfig}
\usepackage{amsfonts}
\usepackage{amsmath}
\usepackage[utf8]{inputenc}
\usepackage[T1]{fontenc}
\usepackage{graphicx}
%\usepackage[english]{babel}
\usepackage[algoruled]{algorithm2e}

\renewcommand{\theequation}{\thesection.arabic{equation}}

\renewcommand{\thefigure}{\thesection.\arabic{figure}}



\renewcommand{\vec}[1]{\mathbf{#1}}
\renewcommand{\theequation}{\thesubsection.\arabic{equation}}
\DeclareGraphicsExtensions{.pdf,.png,.jpg, .gif}

\usepackage{amsthm}

\usepackage{mathtools}

%\usepackage[OT2,T1]{fontenc}
%\DeclareSymbolFont{cyrletters}{OT2}{wncyr}{m}{n}
%\DeclareMathSymbol{\sha}{\mathalpha}{cyrletters}{"58}

\DeclareFontFamily{U}{wncy}{}
\DeclareFontShape{U}{wncy}{m}{n}{<->wncyr10}{}
\DeclareSymbolFont{mcy}{U}{wncy}{m}{n}
\DeclareMathSymbol{\Sh}{\mathord}{mcy}{"58} 
\DeclareMathOperator*{\argmin}{arg\,min}
\DeclareMathSymbol{\kronecker}{\otimes}

\newcounter{eqn}
\renewcommand*{\theeqn}{\alph{eqn})}
%\newcommand{\num}{\refstepcounter{eqn}\text{\theeqn}\;}

\makeatother
\newcommand{\vectornorm}[1]{\left|\left|#1\right|\right|}
\newcommand*\conjugate[1]{\bar{#1}}

\theoremstyle{plain}
\newtheorem{thm}{Theorem}[chapter]
\newtheorem{defn}{Definition}
 %\theoremstyle{plain}
  \newtheorem{theorem}{Theorem}[section]
  \newtheorem{corollary}[theorem]{Corollary}
  \newtheorem{proposition}[theorem]{Proposition}
  \newtheorem{lemma}[theorem]{Lemma}
\newtheorem{example}[theorem]{Example}
  \newtheorem{definition}[theorem]{Definition}
  \newtheorem{conj}[theorem]{Conjecture}
 \newtheorem{condition}{Condition}
 \newtheorem{remark}[theorem]{Remark}

\newcommand{\supp}{\operatorname{supp}} 
\newcommand{\vc}[1]{{\mathbf{ #1}}}
\newcommand{\tn}{\widetilde{\nabla}_{n} }
\newcommand{\Z}{{\mathbb{Z}}}
\newcommand{\re}{{\mathbb{R}}}
\newcommand{\II}{{\mathbb{I}}}
\newcommand{\ep}{{\mathbb{E}}}
\newcommand{\pr}{{\mathbb{P}}}
\newcommand{\FF}{{\mathcal{F}}}
\newcommand{\TT}{{\mathcal{T}}}
\newcommand{\phin}{\phig{n}}
\newcommand{\phig}[1]{\phi^{(#1)}}
\newcommand{\ol}[1]{\overline{#1}}
\newcommand{\eff}{{\rm eff}}
\newcommand{\suc}{{\rm suc}}
\newcommand{\tends}{\rightarrow \infty}
\newcommand{\setS}{{\mathcal{S}}}
\newcommand{\setP}{{\mathcal{P}}}
\newcommand{\setX}{{\mathcal{X}}}
\newcommand{\nec}{{\rm nec}}
\newcommand{\bd}{{\rm bd}}
\newcommand{\kronecker}{{\otimes}}

\begin{document}
\title{Cramer Rao bound for CS}
\maketitle

\section{Introduction}
This document is a short derivation of theCramer Rao bound for the Compressive sensing.

\section{Model and Derivation}

We capture a deterministic sparse vector \(x \in \re^n\)  through a sensing matrix \(A \in \re^{m \times n}\), giving us compressive measurements \(y \in \re^m\).

\begin{equation}
y = Ax
\end{equation}

We can model the signal with:

\begin{equation}
p \left(y \mid x \text{,} \sigma^2 \right) = (2 \pi \sigma^2)^{-K/2} \exp{\left(- \frac{1}{2 \sigma^2} \vectornorm{y - Ax}_{2}^{2} \right)}\exp{(-\lambda\vectornorm{x}_1)} 
\end{equation}

Up to a constant we find that":

\begin{equation}
-\log{p \left(y \mid x \text{,} \sigma^2 \right) }=\frac{1}{2\sigma^2}\vectornorm{Ax-y}_2^2 + \lambda\vectornorm{x}_1
\end{equation}
\label{logprob}

and we calculate that:

\begin{equation}
\frac{\partial^2}{\partial^2x}-\log{p \left(y \mid x \text{,} \sigma^2 \right) } = \frac{1}{\sigma^2}A^T A
\end{equation}
\label{Fisher}

\begin{remark}
The second term in \eqref{logprob} doesn't add anything to this as the first derivative is either \(\lambda\), \(-\lambda\) or \(0\), depending on the sign of \(x\).
\end{remark}

Taking the expectation of \eqref{Fisher} we find that:

\begin{equation}
\ep{\frac{\partial^2}{\partial^2x} -\log{p \left(y \mid x \text{,} \sigma^2 \right) }} = \frac{1}{\sigma^2 n}
\end{equation}

where we have used:

\begin{theorem}[Expected Value of Wishart Matrices]\label{thm:wishart-mean}
Given a matrix  \(W \in \re^{r \times r}\)
\begin{equation}
\ep\left(W\right) = rI
\end{equation}
\end{theorem}

So we find that the Cramer Rao bound is:

\begin{align}
\ep{\vectornorm{\hat{x}-x}} &\geq \frac{\sigma^2}{n}
\end{align}

where \(\hat{x}\) is any unbiased estimator of \(x\).

\end{document}