\section{Distributed Sensing Model}\label{sec:sensingmodel}


We consider a radio environment with a single primary user (PU) and a network of \(J\) nodes collaboratively trying to sense and reconstruct the PU signal in a fully distributed manner by local communication and regularisation only.

We try to sense and reconstruct a wideband signal, using a network of \(J\) (= 50) nodes placed uniformly at random within the square \(  \left[0,1\right]\times \left[0,1\right] \). 

We consider the frequency domain measurements, formed by each node mixing the signal with a random Gaussian signal \(A_j \in \re^n\). The measurements taken at node \(j\) are:

\begin{equation}
y_j = A_jH_jg + w_j
\label{dist_system}
\end{equation}

where \(H_j \in \re\) is the scalar channel gain, and \(w_j \sim \mathcal{N}(0,\sigma^2_n) \in \re \) is additive white Gaussian noise. 

For the purposes of comparison in section (\ref{sec:results}), this corresponds to the concatenated system:

\begin{equation}
y = AHg + w
\label{system}
\end{equation}

where \(H \in \re^{n \times n}\) is a block diagonal matrix of channel gains.

The system  \ref{system} can then be solved (in the sense of finding the sparse vector \(a\) (\ref{basis}) by convex optimisation via minimising the objective function:

\begin{equation}
\hat{a} = \argmin_{a} \frac{1}{2}\|AHL^{T}a-y\|_2^2 + \lambda \|a\|_1
\label{opt}
\end{equation}

where \(\lambda\) is a parameter chosen to promote sparsity. Larger \(\lambda\) means sparser \(a\).